\documentclass[./body]{subfiles}
\begin{document}

\section{独自定義の環境}

\subsection{\texttt{hmq}環境(演習問題用)}

\begin{hmq}\label{vec:scq}
以下のものをベクトル量とスカラー量に分類せよ。

\noindent
\phantom{.}\hfill{速度}\hfill{時間}\hfill{体積}
\hfill{動画の再生数}\hfill{密度}\hfill\phantom{.}
\end{hmq}

\subsection{\texttt{hmans}コマンド(演習問題の解答用)}

\hmans{vec:scq}
\textsf{ベクトル量}: 速度、\textsf{スカラー量}: 時間、体積、動画の再生数、密度

\subsection{\texttt{hmbox}環境(定理など)}

\begin{hmbox}{差を求める式}
\[\text{対象}-\text{基準}=\text{差}\label{vec:diff2}\]
\end{hmbox}

\subsection{\texttt{hmcomp}環境(補足など)}

\begin{hmcomp}{接線の傾き}
$\Delta t$を限りなく$0$に近づけることは、接線の傾きを求めることにも対応しています。
\end{hmcomp}

\subsection{\texttt{hmemph}コマンド(強調)}

\hmemph{瞬間の速さは接線の傾きに対応しています}。

\subsection{\texttt{途中} コマンド}

\途中

\subsection{数式の例}

\verb+\hmVEC+ で、始点終点を記したベクトルを表現(\verb+\protect+ 付き)。

\[\hmVEC{OX}\]

\verb+\hmvec+ でベクトルを表現(太字)。

\begin{align*}
&\lim_{\Delta t\to0}\left[\frac{A[\hmvec{x}[t+\Delta t],t+\Delta t]-A[\hmvec{x}[t],t]}{\Delta t}\right]\\
=&\sum_i \frac{\partial A}{\partial x_i}\frac{\partial x_i}{\partial t}+\frac{\partial A}{\partial t}\\
=&\left(\frac{\partial}{\partial t}+\hmvec{u}\cdot\nabla\right)A=\frac{DA}{Dt}
\end{align*}


\section{留意事項}

\begin{itemize}
\item \texttt{subfiles}\footnote{\url{https://qiita.com/sankichi92/items/1e113fcf6cc045eb64f7}}
	を使っています。部ごとにディレクトリを分けて、章ごとにファイルを分ける方針で。
\item fonts フォルダのフォントをインストールしてください。
\item 和文の太字が効かないので、どうにかして対処する。
\item タイプライタ体(beramono)の太字も効いていない。
	(そもそも TU/fvm/b/n が存在しないので、TU/fvm/m/n で代用するように設定する)
\end{itemize}

\end{document}